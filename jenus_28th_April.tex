%\texttt{}% ****** Start of file aipsamp.tex ******
%
%   This file is part of the AIP files in the AIP distribution for REVTeX 4.
%   Version 4.1 of REVTeX, October 2009
%
%   Copyright (c) 2009 American Institute of Physics.
%
%   See the AIP README file for restrictions and more information.
%
% TeX'ing this file requires that you have AMS-LaTeX 2.0 installed
% as well as the rest of the prerequisites for REVTeX 4.1
%
% It also requires running BibTeX. The commands are as follows:
%
%  1)  latex  aipsamp
%  2)  bibtex aipsamp
%  3)  latex  aipsamp
%  4)  latex  aipsamp
%
% Use this file as a source of example code for your aip document.
% Use the file aiptemplate.tex as a template for your document.

\documentclass[aip,twocolumn,superscriptaddress,showpacs,amsmath]{revtex4}
%%%%%%%preprint,amssymb,%%%%%%%%%%
\usepackage{graphicx}
\usepackage[mathlines]{lineno}
\begin{document}

\bibliographystyle{apsrev}

%\title{{Proton acceleration enhanced by reusing prepulse Proton energy enhancement
\title{{ Nearly Critical Dense Plasmas generated by laser ablation for efficient proton acceleration
}}

\author{S. Zhao}
\affiliation{School of Physical Sciences, University of Chinese
Academy of Sciences, Beijing 100191, China. } \affiliation{State Key
Laboratory of Nuclear Physics and Technology,and Key Lab of High
Energy Density Physics Simulation, CAPT, Peking University, Beijing
100871, China. }
\author{C. Lin}\email[]{linchen0812@pku.edu.cn}
\affiliation{State Key Laboratory of Nuclear Physics and
Technology,and Key Lab of High Energy Density Physics Simulation,
CAPT, Peking University, Beijing 100871, China. }

\author{X.T.He}
\affiliation{State Key
Laboratory of Nuclear Physics and Technology,and Key Lab of High
Energy Density Physics Simulation, CAPT, Peking University, Beijing
100871, China. }


\author{J. E. Chen}
\affiliation{School of Physical Sciences, University of Chinese
Academy of Sciences, Beijing 100191, China. } \affiliation{State Key
Laboratory of Nuclear Physics and Technology,and Key Lab of High
Energy Density Physics Simulation, CAPT, Peking University, Beijing
100871, China. }
\author{X.\,Q. Yan} \email[]{x.yan@pku.edu.cn}

\affiliation{State Key Laboratory of Nuclear Physics and
Technology,and Key Lab of High Energy Density Physics Simulation,
CAPT, Peking University, Beijing 100871, China. }




\date{\today}

\begin{abstract}


Near Critical Dense (NCD) plasma is an optimal medium for high efficiency
laser plasma energy coupling. We propose to use controllable ablation pulse
for adjustable profile NCD generation, helping enhance the laser ion
acceleration. The matching condition between ablation and CPA main pulse are
discussed for optimal enhancement. PIC simulations confirm it and show
that hundreds MeV proton beam can be achieved by hundreds of TW laser.

\end{abstract}
\pacs{52.38.Kd, 41.75.Jv, 52.35.Mw, 52.59.-f}

\maketitle




%%%%%%%%%%%%%%%%%%%%%%%%%%%%%%%%%%%%%%%%%%%%%%%%%%
%general description of the prepulse and the effect
%%%%%%%%%%%%%%%%%%%%%%%%%%%%%%%%%%%%%%%%%%%%%%%%%%
Underdense plasmas and solid density plasmas are extensively used in laser
plasma interactions. Recently many novel plasma phenomena were discovered when
ultra-intense laser pulse interacting with nearly critical dense plasmas ($0.1
n_c<n_e<\gamma n_c$). For example, proton shock acceleration from critical
dense gas target [nature physics, UCLA] ,plasma self-focusing [], plasmas
lens[], resonnant electron acceleration and etc.

It's known that the plasma expansion at the rear surface is harmful for rear
surface ion acceleration \cite{Roth} while the front ablation can be positive
in case of 'proper' control on plasma parameters. In experiments the
'controllable' ablation preplasmas can be achieved by an individual pulse with
moderate intensity($10^{12}W/{cm}^2 ~ 10^{14}W/{cm}^2$) and long duration(100s
ps). The ablation plasma can be controlled in somewhat by adjusting the pulse
intensity, duration and the time delay between main CPA pulse. It's reported
that $25\%$ proton cutoff energy improvement by P.Mckenna et
al.\cite{MCKENNA}, also $10\%$ by Y.Glinec et al.\cite{Glinec}. Recently it is
also reported \cite{Wang} and Sgattoni et al.\cite{Sgat} that double layer
target (NCD+foil) can increase electrons temperature and the ion energy  and
homogeneous NCD layer is assumed in their simulations, however, how to
generate such NCD+foil target is still challenging for sapphire laser so far.

In this paper we propose to recycle the usually harmful prepulse
(including ASE, pedestal and so on) as helpful target ablation source and
discuss the optimization for the ion acceleration enhancement due to the
ablation. The dense energetic electrons generated in ablation plasma is
responsible for the high energy ion output. However the electron beam quality
is vulnerable during generation and transport. To make sure the effective
generation and transportation of the electron source, matching of CPA pulse
and the plasma density profile is derived, which is in good agreement with simulations.
And for given laer system, the highest energy ion beam can
be achieved by manipulating prepulse 
. It's in capable of some laboratories. 
%%%%%%%%%%%%%%%%%%%%%%%%%%%%%%%%%%%%%%%%%%%%%%%%%%%%%%%%%%%%

%\subsection{target ablation}
\begin{figure}[htbp]
\includegraphics[width=0.45\textwidth]{fig_1}
\caption{\label{fig1}(Color online) schematic graphic. CPA pulse propagates in
the preplasma, generating energetic electrons. These electrons
contribute to the further ion acceleration enhancement.}
\end{figure}

%%%%%%%%%%%%%%%%%%%%%%%%%%%%added in Feb 24th, 2014%%%%%%%%%%%%%%%%%%%%%%%%%%
In laser driven ablation, laser energy is absorbed mainly at or outside the
critical density surface. And the ablation area is divided into two parts
according to the density, under and over critical. In under dense region the
plasma expansion to the vacuum is assumed to be isothermal. The 1D
self-similar model of plasma expansion for an exponential density
profile\cite{Max}, $n_e=n_c exp(-(x-x_0)/{c_s(t-t_0)})$, where $n_c$ is he
critical density and $c_s=\alpha \sqrt{\gamma z^*T_e/\mu}$ in $cm/s$ is the
adiabatic sound velocity. $\alpha$ is a factor determined by target material,
and for aluminum $9.79 \times 10^5$, $\gamma$ the adiabatic constant, $\mu$
the atomic number, $Z^*$ the ionization state, $x_0$ and $t_0$ the initial
front surface position and time respectively . $T_e$ is the electron
temperature in eV, which in classical model is typically given by
$T_e(eV)=10^{-6} (I_L{\lambda}^2)^{2/3}$, where $I_L$ is the ablation laser
intensity.


In the dense area, energy absorbed at the critical surface is transported
inward the target by electronic conduction. The ablation front is accompanied
with the pressure wave in acoustic velocity. For the ablative area between the
ablation front and critical surface, the density distribution is derived from
the steady state model. The dense area distance(from criticalsurface to ablation
front) $x_0 \propto  I_{prepulse}^{4/3} {\lambda}^{14/3}$\cite{man,mora2},
and the density profile has no analytical expression.



A simulation for typical target ablation is given by hydrodynamic code
MULTI1D\cite{multi}. 2.5 $\mu$m 1.9 $g/cm^3$Al foil is ablated by 200ps,
$10^{12} \text{w}/cm^2$ flattop ablation pulse(corresponding to 5mJ energy).
After 200ps, the 1D density profile along the laser propagation direction is
in Fig~\ref{fig2}, the pink area represents the origin target, the black line
for the ablated target. The critical density surface at $x=30 \mu m$ divides
the under dense and dense area. The density compression is available at the
ablation front due to heating pressure. The plasma properties, such as:
density distribution, electrons temperature et al, will be loaded as initial
PIC parameters.


\begin{figure}[htbp]
\includegraphics[width=0.45\textwidth]{fig_2}
\caption{\label{fig2}(Color online) MULTI1D simulation result of the plasma
density profile along the laser axis. The initial target is the pink block.
After 200 ps ablation by $10^{12} W/{cm}^2$ pulse, the target is composed of
under dense, NCD, foil part. }
\end{figure}






%%%%%%%%%laser guide is important%%%%%%%%%%%%%%%%

When CPA pulse propagates in the under dense pre-plasma, it
channels\cite{Chanel} and self-focuses. Electrons in front of the laser pulse
are expelled and drift along the channel wall. The charge separation and
electrons longitudinal motion lead to quasi-static electric and magnetic
fields, in the channel some laser-recaptured electrons undergo betatron
oscillation. Once the laser frequency experienced by the electrons matches the
electron oscillation frequency, direct laser acceleration dominates, leading
to electron bunches with few times higher temperature comparing with
ponderomotive heating\cite{Pukhov,Liu}. Meanwhile the electron bunches are
tightly focused by the quasi-static fields.




The energetic dense electron bunches have the advantages of both temperature and density
comparing with the laser-foil heating. So it is crucial to transfer as much
laser energy to these electrons and keep the beam quality for the future ion acceleration.
However the beam quality can
be easily violated during transportation. For example, once the electrons are out of
the channel, the focusing forces will disappear. Electron bunches
dispersion outside the channel will be surely serious and unavoidable, leading
to lower electron energy density and consequently poor output ion beam
quality. To guarantee the further acceleration, it is imperative to keep the
electron bunches being focused in the channel until reaching the foil target.
The ideal case is that, under the pre-condition of channel connect with the
foil target, electrons achieve the most energy gain. The laser pulse functions
as the energy source and channel driller, exhausted while just reaching the
foil target. Then high quality electrons with the highest temperature and
density, interact directly with the foil target for ion acceleration.Then 
it arrives at the point that, laser depletion distance should match the
pre-plasma expansion profile. Laser depletion distance for constant density is
investigated by L. Willingale et al\cite{wil}. Generally, for low density
plasma the laser depletion distance is
\begin{equation*} d=2c{\tau}_Ln_c /n_e \end{equation*}
while for the dense,
\begin{equation*} d=a_0 c \sqrt{({m_e n_c}/{2 m_i n_e})} {\tau}_L \end{equation*}
where ${\tau}_L$ is the laser duration, $n_e$ plasma density, $a_0$ normalized
laser intensity, $m_i$ is the ion mass. For relativistic laser pulse, under
critical plasma is nearly transparency, so the laser depletion is almost  inside the dense
area(over critical density area). In our case, laser propagates in both the
under critical and dense area. Since the relativistic laser can get through
the under critical part, no matter how long it is. In dense area, the
estimation reads
\begin{equation*} D_{depletion} = \alpha ({\tau}_{prepulse}, I_{prepulse}) a_0{\tau} ,\end{equation*}

where $\alpha({\tau}_{prepulse}, I_{prepulse})$ is factor. Then we equals it
to the dense area distance,
\begin{equation*} x_0 \propto {I_{prepulse}}^{4/3} {\lambda}^{14/3} \end{equation*}
it arrives the relationship between the prepulse and CPA pulse that

\begin{equation*} a_0{\tau} \propto \alpha({\tau}_{prepulse}, I_{prepulse}) {I_{prepulse}}^{4/3}
{\lambda}^{14/3} \end{equation*}.


To simplify it, we have a constant $\alpha({\tau}_
{prepulse},I_{prepulse})$ for a given pre-pulse, finally
\begin{equation*} a_0{\tau} \propto {I_{prepulse}}^{4/3} {\lambda}^{14/3}
\end{equation*}.




%%%%%%%%%%%%%%%%%%%%%%%%%%%%%%%%%%%%%%%%%%%%%%%%%%%%%%%%%%%%%%%%%%%%%%%%%%%%
%%%%%%%%%%%% add in 13 March for the fifth time %%%%%%%%%%%%%%%%%%%%%%%%%%


Relativistic PIC simulation code KLAP\cite{KLAP,Yan2008} is used to verify
these theoretical analysis. In PIC simulation length, time, charge, mass and
density are normalized to laser wavelength ${\lambda}_0=1.06\mu $m, laser
period $T=3.3fs$, electron charge e, electron mass $\text{m}_e$ and the
critical density $\text{n}_c=m_e{\varepsilon}_0{{\omega}_0}^2/e^2$,
respectively. The aluminum target is ablated by 200ps prepulse, and the
properties are loaded from the hydrodynamic simulation, also attached with
$10\text{n}_c$ $0.1\mu $m H layer at the rear surface as contamination in
actual. The CPA pulse is p-polarized while the polarization plane is in the
simulation plane and has Gaussian distribution in both space and time with the
normalized vector potential $a=11.0$ for peak intensity (corresponds to
$I/{{\lambda}_0}^2=1.67{\times}10^{20}\text{W}/\text{{cm}}^2{\mu m}^2$), where
the pulse duration FWHM is 9T(30fs), the focal radius
${\sigma}_0=5{\lambda}_0$, total energy of 5J. The size of the simulation box
is $80{\lambda}_0\times 40 {\lambda}_0$. Here, we use unique $3200\times1600$
grids with space resolution of $\delta x = \delta y = {\lambda}_0/40$. The
time step is $0.01T$, simulation begins at 0T and stops at 200T to make sure
that all the acceleration proceses are in view.


\begin{figure}[htbp]
\includegraphics[width=0.45\textwidth]{fig_3a}
\vspace{2.00mm}
\includegraphics[width=0.45\textwidth]{fig_3b}
\vspace{2.00mm}
%\includegraphics[width=0.45\textwidth]{fig_3c}
\includegraphics[width=0.45\textwidth]{fig_3d}
\caption{\label{fig3}(Color online) laser propagation at t=20, 50, 120T 
respectively in (a) ; electrons energy density in near critical plasma in
(b), the time corresponding to laser propagation. The ion channel at t=120T
are formation in (c).}
\end{figure}



In the near critical plasma the laser propagates and transfers energy, especially 
to the electrons.
The laser pulse evolution and electron information snapshot are demonstrated
in Fig~\ref{fig3}. The laser pulse are focused to 2um radius spot at $t=50T$,
and exhausted at $t=120T$, the snapshot of the laser electric field Ey is
depicted in Fig~\ref{fig3}.(a), the intensity improves several times higher.
Correspondingly energetic electron bunches are generated inside the preplasma, 
illustrated in energy density distribution Fig~\ref{fig3}.(b), where the 
red bunches show evident difference than the backgroud.
The channel formation is shown in Fig~\ref{fig3}.(c), directly connect to the target so that
the highly focused electrons can reach the foil target with kept energy density.



%%%%%%%%%%%%%%%%%%%%%% add a section for the statistics %%%%%%%%%%%%%%%%%%
%%%%%%%%%%%   of the laser reflection  will be suitable to get   %%%%%%%%%
%%%%%%%%%%%%%%%%%%%%%%%%%%%%%%%%%%%%%%%%%%%%%%%%%%%%%%%%%%%%%%%%%%%%%%%%%%


\begin{figure}[htbp]
\includegraphics[width=0.230\textwidth]{fig_5a}
\vspace{2.00mm}
\includegraphics[width=0.230\textwidth]{fig_5b}
\vspace{2.00mm}
\includegraphics[width=0.230\textwidth]{fig_5e}
\vspace{2.00mm}
\includegraphics[width=0.230\textwidth]{fig_5f}
\includegraphics[width=0.230\textwidth]{fig_5g}
\includegraphics[width=0.230\textwidth]{fig_3d}
\includegraphics[width=0.230\textwidth]{fig_5c}
\vspace{2.00mm}
\includegraphics[width=0.230\textwidth]{fig_5d}
\includegraphics[width=0.450\textwidth]{fig_5h}
\caption{\label{fig4}(Color online) the laser propagation evolution for longer
and shorter case in (a,b); the electron density in (c,d) and channel formation
in(e,f)}
\end{figure}




As discussed in last section that the laser pulse should match the prepulse,
here we have both 'longer' and 'shorter' reference simuation to show the effect
on energetic electron beam properties and output ion beam quality. In 'longer' case, 
280ps prepulse is used for target ablation, with preplasma expansion distance 60$\mu$m.
In 'shorter' case, 100ps prepule and 20$\mu$m preplasma. Comparings 
between the laser, electron information are in Fig~\ref{fig4}. 
The laser propagations are in Fig~\ref{fig4}(a,b), the pulse is totally absorbed half way to
target in 'longer' case while strongly reflected in 'shorter' case. For electron
density distribution in Fig~\ref{fig4}(c,d), both are lower than the 'optimal'
case. The 'longer' case has the scattering problom while the 'shorter' hasn't achived 
enough energy from the laser. The ion channels formation are In Fig~\ref{fig4}(c,d), it is
evident that the channel issue accounts for the electrons beam expansion in the 'longer'
case. To  
The spectra of electrons inside preplasma in Fig~\ref{fig4}(e) for both 'optimal', 
'longer' and 'shorter' cases. In the 'optimal' and 'longer' cases, lasers energy are totally absorbed 
inside the preplasma, both their electron temperature reach nearly the same maximum. 
For the 20$\mu$m case, electron heating inside the preplasma is far from enough. When we track the 
electrons from the target in Fig~\ref{fig4}(f), things are reversed that the 'shorter' has advantages 
mainly because there is still laser on the target to heat the electrons inside. However accounting for 
the energetic electrons which will contribute to 
the ion acceletion enhancement, the 'short' is in disadvantages.
Clearly if plasma expansion distance is shorter, the laser pulse energy
coupling to the plasma is not enough, the electrons energy are far 
from the maximum. Otherwise if the plasma
expansion is too much, the electrons beam in preplasma achieve the highest temperature, 
however the beam quality will be deteriorated during the transportation. Finally we arrive at
the output proton spectra in Fig~\ref{fig4}(g). Only the 'matching' is the optimal, and 
for too 'short' or 'long', the enhancements would discount and even be nagetive.





\begin{figure}[htbp]
\includegraphics[width=0.50\textwidth]{fig_6}
\caption{\label{fig6}(Color online) the optimal ablation pulse
matching the CPA pulse. the black line represents for the data get
by the estamation formula while the square for the simulation
result}
\end{figure}





%%%%%%%%%%%%%%%%%%%%%%%%%%%%%%%%%%%%%%%%%%%%%%%%%%%%%%%%%%%%%%%%%
%%%%%%%%%      new section added in 2013-10-08        %%%%%%%%%%%
%%%%%%%%%%%%%%%%%%%%%%%%%%%%%%%%%%%%%%%%%%%%%%%%%%%%%%%%%%%%%%%%%


\subsection{conclusion}
In this paper, we investigated the enhanced ion acceleration by ablation
pulse. It causes target expansion, leading to pre-plasma + foil
double-layer structure. Pre-plasma expansion scale is determined by
the pulse intensity and duration. When CPA pulse interacts with the structure,
the output ion beam quality can be improved due to the energetic electrons generated in pre-plasma. 
We get a scaling law between the CPA pusle and ablation pulse, that for
given CPA pulse, the 'optimum' enhancement can be achieved by
adjusting the ablation pulse duration. Simulations show several times improve in ion cutoff energy.
Hopefully that our work will
apply to the ion acceleration experiment since the laser and target
properties are realizable in experiment.



\begin{acknowledgments}
This work was supported by National Natural Science Foundation of
China (Grant Nos. 11025523,10935002,10835003,J1103206) and National
Basic Research Program of China (Grant No. 2011CB808104).
\end{acknowledgments}


%\nocite{*}
%\bibliography{bib1}% Produces the bibliography via BibTeX.

\begin{thebibliography}{28}
\bibitem{radiography} M. Borghesi1, D. H. Campbell, A. Schiavi, M. G. Haines, O. Willi, A. J. MacKinnon, P. Patel, L. A. Gizzi, M. Galimberti, R. J. Clarke, F. Pegoraro, H. Ruhl, S. Bulanov, Phys. Plasmas 9, 2214 (2002).
\bibitem{trerapy}T. Zh. Esirkepov, S. V. Bulanov, K. Nishihara, T. Tajima, F. Pegoraro, V. S. Khoroshkov, K. Mima1, H. Daido, Y. Kato, Y. Kitagawa, K. Nagai, S. Sakabe, Phys.Rev. Lett. 89, 175003 (2002).
\bibitem{tnsa}H. Schwoerer, S. Pfotenhauer, O. Jackel, K. U. Amthor, B. Liesfeld, W.
Ziegler, R. Sauerbrey, K. W. D. Ledingham, T. Esirkepov, Nature 439, 445
(2006);

R. A. Snavely, M. H. Key, S. P. Hatchett, T. E. Cowan, M. Roth, T. W.
Phillips, M. A. Stoyer, E. A. Henry, T. C. Sangster, M. S. Singh, S. C. Wilks,
A. MacKinnon, A. Offenberger, D. M. Pennington, K. Yasuike, A. B. Langdon, B.
F. Lasinski, J. Johnson, M. D. Perry, and E. M. Campbell, Phys. Rev. Lett. 85,
2945 (2000).

B. M. Hegelich, B. J. Albright, J. Cobble, K. Flippo, S. Letzring,
M. Paffett, H. Ruhl, J. Schreiber, R. K. Schulze, J. C. Fernandez,
Nature 439, 441 (2006);



Stephen P. Hatchett, Curtis G. Brown, Thomas E. Cowan, Eugene A. Henry, Joy
S.Johnson, Michael H. Key, Jeffrey A. Koch, A. Bruce Langdon, Barbara F.
Lasinski, Richard W. Lee, Andrew J. Mackinnon, Deanna M. Pennington, Michael
D. Perry, Thomas W. Phillips, Markus Roth, T. Craig Sangster, Mike S. Singh,
Richard A. Snavely, Mark A. Stoyer, Scott C. Wilks, Kazuhito Yasuike ,Phys.
Plasmas 7, 2076 (2000);


\bibitem{rpa} Andrea Macchi, Silvia Veghini, and Francesco Pegoraro,
Phys. Rev. Lett. 103, 085003 (2009);

 T. Esirkepov, M. Borghesi, S. V. Bulanov, G. Mourou,
T. Tajima, Phys. Rev. Lett. 92,175003 (2004);

A. Henig, D. Kiefer, K. Markey, D. C. Gautier, K. A. Flippo, S.
Letzring, R. P. Johnson, T. Shimada, L. Yin, B. J. Albright, K. J.
Bowers, J. C. Fernandez, S. G. Rykovanov, H. C. Wu, M. Zepf, D.
Jung, V. Kh. Liechtenstein, J. Schreiber, D. Habs, B. M. Hegelich,
Phys. Rev. Lett. 103, 045002 (2009);



\bibitem{Roth} M. Roth, A. Blazevic, M. Geissel, T. Schlegel, T. E. Cowan,
M, Allen, J. C. Gauthier, P, Audebert, J. Fuchs, J. Meyer-ter-Vehn, M.
hegelich, S. Karsch, and A. Pukhov, Phys.Rev.S.T.A.B. 5,061301 (2002).

\bibitem{MCKENNA} P. Mckenna, D.C. Carroll, O. Lundh, F. Nurnberg, K.
Markey, S. Bandyopadhyay, D. Batant, R. G. Evans, R. Jafer, S. Kar, D. Neely,
D. Pepler, M. N. Quinn, R. Redaelli, M. Roth, C. -G. Wahlstrom, X. H. Yuan,
and M. Zepf, Laser and Particle Beams, 26, 591 (2008);

\bibitem{Glinec} Y.Glinec, G.Genoud, O.Lundh, A.Persson, C.G.
Wahlstrom, Applied Physics B, 93, 317(2008)


\bibitem{Wang} H. Y. Wang, C. Lin, F. L. Zheng, Y. R. Lu, Z. Y. Guo,
X. T. He, J. E. Chen, and X. Q. Yan, Phys. Plasmas 18, 093105 (2011)

\bibitem{Sgat} A. Sgattoni, P. Londrillo, A. Macchi, and M. Passoni, Phys. Rev. E 85, 036405 (2012).

\bibitem{xps} A. Jullien, X. Chen, A. Ricci, J. P. Rousseau, R. Lopez-Martens,
L. P. Ramirez, D. Papadopoulos, A. Pellegrina, F. Druon, P. Georges,
International Conference on Ultrahigh Intensity Laser (2010)


\bibitem{Max} C. E. Max, C. F. McKee and W. C. Mead, Phys. Fluids 23, 1620 (1980).

\bibitem{man} W. M. Manheimer, D. G. Colombant, and J. H. Gardner, Phys.
     Fluids 25, 1644 (1982).

\bibitem{mora2} P. Mora, Phys. Fluids 25, 1051 (1982).

\bibitem{multi} R. Ramis, R. Schmalz, J. Meyer-Ter-Vehn MULTI-A
computer code for one-dimensional radiation hydrodynamics, Comp. Phys. Comm.
49, 475(1988)

\bibitem{Chanel}  M. Borghesi, A. J. MacKinnon, L. Barringer, R. Gaillard, L. A.
Gizzi, C. Meyer, and O. Willi, A. Pukhov and J. Meyer-ter-Vehn, Phys. Rev.
Lett 78, 879 (1997)



\bibitem{Pukhov} A. Pukhov, J. Meyer-ter-Vehn, Phys. Rev. Lett 76, 3975 (1996).
 A. Pukhov, Z. M. Sheng, J. Meyer-ter-Vehn , Phys.Plasmas 6, 2847 (1999).

\bibitem{Liu} B. Liu, H. Y. Wang, J. Liu, L. B. Fu, Y. J. Xu, X. Q. Yan, and X. T. He, Phys. Rev. Lett 110,
045002 (2013)

%\bibitem{Gahn} C. Gahn, G. D. Tsakiris, A. Pukhov, J. Meyer-ter-Vehn, G. Pretzler, P. Thirolf, D. Habs, K. J. Witte, Phys. Rev. Lett 83, %4772 (1999).



\bibitem{Mora} P. Mora, Phys. Rev. Lett 90, 185002 (2003).









\bibitem{wil}L. Willingale, S. R. Nagel, A. G. R. Thomas, C. Bellei, R. J. Clarke, 
A. E. Dangor, R. Heathcote, M. C. Kaluza, C. Kamperidis, S. Kneip, 
K. Krushelnick, N. Lopes, S. P. D. Mangles, W. Nazarov, P. M. Nilson, and
Z. Najmudin, Phys. Rev. Lett. 102, 125002 (2009).




\bibitem{KLAP}Zheng-Ming Sheng, Kunioki Mima, Jie Zhang, Heiji Sanuki, Phys. Rev. Lett 94, 095003 (2005).

\bibitem{beg} F. N. Beg, A. R. Bell, A. E. Dangor, C. N. Danson, A. P. Fews, M. E.
Glinsky, B. A. Hammel, P. Lee, P. A. Norreys, and M. Tatarakis,
Phys. Plasmas 4, 447(1997).

\bibitem{Mora} P. Mora, Phys. Rev. Lett 90, 185002 (2003).

\end{thebibliography}
\end{document}


%
% ****** End of file aipsamp.tex ******
